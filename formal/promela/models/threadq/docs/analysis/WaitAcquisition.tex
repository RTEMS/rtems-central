\section{Wait Acquisition}\label{sec:waitacq}

Here we unfold function calls in \verb"_Thread_Wait_acquire_critical"
to expose its behaviour.

For now, here is the given code:
\begin{nicec}
RTEMS_INLINE_ROUTINE void _Thread_Wait_acquire_critical(
  Thread_Control       *the_thread,
  Thread_queue_Context *queue_context
)
{
  Thread_queue_Queue *queue;

  _Thread_Wait_acquire_default_critical(
    the_thread,
    &queue_context->Lock_context.Lock_context
  );

  queue = the_thread->Wait.queue;
  queue_context->Lock_context.Wait.queue = queue;

  if ( queue != NULL ) {
    _Thread_queue_Gate_add(
      &the_thread->Wait.Lock.Pending_requests,
      &queue_context->Lock_context.Wait.Gate
    );
    _Thread_Wait_release_default_critical(
      the_thread,
      &queue_context->Lock_context.Lock_context
    );
    _Thread_Wait_acquire_queue_critical( queue, &queue_context->Lock_context );

    if ( queue_context->Lock_context.Wait.queue == NULL ) {
      _Thread_Wait_release_queue_critical(
        queue,
        &queue_context->Lock_context
      );
      _Thread_Wait_acquire_default_critical(
        the_thread,
        &queue_context->Lock_context.Lock_context
      );
      _Thread_Wait_remove_request_locked(
        the_thread,
        &queue_context->Lock_context
      );
      _Assert( the_thread->Wait.queue == NULL );
    }
  }
}

\end{nicec}

\newpage
\subsection{Semantics of TWadc}

\verb"_Thread_Wait_acquire_default_critical"

Unfolded:
\begin{nicec}
RTEMS_INLINE_ROUTINE void _Thread_Wait_acquire_default_critical(
  Thread_Control   *the_thread,
  ISR_lock_Context *lock_context
)
{
  // _ISR_Get_level() != 0
  {
    (void) lock_context;
    {
      unsigned int my_ticket;
      unsigned int now_serving;

      my_ticket =
        _Atomic_Fetch_add_uint(
           & ( & ( & ( & the_thread->Wait.Lock.Default )->Lock )->Ticket_lock )->next_ticket,
           1U,
           ATOMIC_ORDER_RELAXED
        );
      do {
        now_serving =
          _Atomic_Load_uint(
            &(&(&( &the_thread->Wait.Lock.Default )->Lock)->Ticket_lock)->now_serving,
            ATOMIC_ORDER_ACQUIRE
          );
      } while ( now_serving != my_ticket );
    }

  }
}
\end{nicec}
We note that \verb"Lock" and \verb"Ticket_lock" are the sole field
in their given struct, and so can be considered the pointer equivalent
of a no-op.
This means that (given the obvious shorthand):
\begin{eqnarray*}
   \& p \to Lck  &=& p
\\ \& q \to Tlck &=& q
\end{eqnarray*}

So we can simplify the long pointer expressions above as follows:
\begin{eqnarray*}
   &      & \& (\& (\&  (\&t \to W.L.D) \to Lck) \to Tlck) \to nt
\\ &\equiv& \& (\& (\&t \to W.L.D) \to Tlck) \to nt
\\ &\equiv& \& (\&t \to W.L.D) \to nt
\\ &\equiv& \& (\&((*t).W.L.D)) \to nt
\\ &\equiv& \& ((*(\&((*t).W.L.D))).nt)
\\ &\equiv& \& ( (*t).W.L.D.nt )
\\ &\equiv& \& t \to W.L.D.nt
\end{eqnarray*}

We get the following code:
\begin{nicec}
RTEMS_INLINE_ROUTINE void _Thread_Wait_acquire_default_critical(
  Thread_Control   *the_thread,
)
{
  // _ISR_Get_level() != 0
  unsigned int my_ticket;
  unsigned int now_serving;

  my_ticket =
    _Atomic_Fetch_add_uint(
       & the_thread->Wait.Lock.Default.next_ticket, // Lock.Ticket_lock
       1U,
       ATOMIC_ORDER_RELAXED
    );
  do {
    now_serving =
      _Atomic_Load_uint(
        & the_thread->Wait.Lock.Default.now_serving, // Lock.Ticket_lock
        ATOMIC_ORDER_ACQUIRE
      );
  } while ( now_serving != my_ticket );
}
\end{nicec}
