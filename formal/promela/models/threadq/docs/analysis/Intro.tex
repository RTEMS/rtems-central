\section{Introduction}

Here we perform an analyis of the RTEMS-SMP Thread Queue implementation
\cite{Burns:2013:MrsP,Catellani:2015:MrsP},
in order to develop Promela models of same.

Our focus is on two functions:
\begin{description}
  \item [Wait Acquisition] \verb"_Thread_Wait_acquire_critical"
    \\ include/rtems/score/threadimpl.h:1795
  \item [Queue Path Acquisition] \verb"_Thread_queue_Path_acquire_critical"
    \\ cpukit/score/src/threadqenqueue.c:262
\end{description}

Code snippets are provided for the above,
and their dependencies,
in Appendix \ref{sec:snippets}.
We also list all the key data-structure definitions in play
in Appendix \ref{sec:datatypes}.

In addition,
we are basing our high-level plan on recent publications regarding MrsP,
two in 2017 by Burns and Wellings
\cite{Zhao:2017:MrsP,Garrido:2017:MrsP},
and a CISTER thesis from 2019 by Gomes
\cite{Gomes:2019:MrsP}.

Perhaps \cite{Zhao:2020:MrsP,Zhao:2021:MrsP}
are relevant?


We plan to use the following key techniques to describe semantics:
\begin{itemize}
  \item Z/UTP-stype predicate calculus (pre/post/rely/guarantee)
  \item Separation Logic
  \item State machines (B\"{u}chi automata)
\end{itemize}

\subsection{Unfolding}

We use the term ``unfolding'' to describe the process of working through
the source code to get the true underlying behaviour.
This involves:
\begin{itemize}
  \item
    Implementing all \verb"#ifdef" and \verb"#ifndef" macros
  \item
    Replacing some function calls with their bodies
  \item
    Expanding code-generating macros.
    This may require:
    \begin{itemize}
      \item
        Replacing any occurrence of \verb"do { code } while ( 0 )"
        by \verb"code".
     \item
       Flattening out deeply left-nested usage of the address operator \verb"&"
    \end{itemize}
\end{itemize}

\subsection{The Plan}

We are going to model the behaviour of MrsP semaphore obtain and release,
with multiple processors, tasks and resources,
over a range of priority allocations.
We take the implementation description in
\cite[Sec VIII, pp288--9]{Burns:2013:MrsP}
as a high-level description of the lock/unlock protocol,
and lower level design from
\cite[Chp. 5, App. 2]{Gomes:2019:MrsP}.
