\section{Atomics}\label{sec:atomics}

Here we perform an analysis of C11 Atomics to support their modelling
and consequent testing.

We will look at the following C11 atomics:
\begin{description}
  \item [Fetch-Add] \verb"obj' = atomic_fetch_add_explicit( obj, arg, order )"
  \item [Load] \verb"obj' = atomic_load_explicit( obj, order )"
  \item [Store] \verb"atomic_store_explicit( obj, desired, order )"
\end{description}
Note that the first two return a revised value for \verb"obj",
while the third has no return value.

The only memory model in scope is total-store order (TSO),
as per the SPARC architecture (sparcv8.pdf).

From \url{https://docs.oracle.com/cd/E23824_01/html/819-3196/hwovr-15.html},
we get:
\begin{quote}
``
TSO guarantees that the sequence in which store, FLUSH,
and atomic load-store instructions appear in memory for a given processor
is identical to the sequence in which they were issued by the processor.
''
\end{quote}
We note that FLUSH instructions are out-of-scope,
as their only relevance seems to be to the use of self-modifying code.

Their relevant usage is as follows:

\begin{tabular}{|c|c|}
  \hline
  Atomic Action & Memory Model(s)
\\\hline
  Load & Relaxed, Acquire
\\\hline
  Store & Relaxed, Release
\\\hline
  Fetch-Add & Relaxed
\\\hline
\end{tabular}

Given that we are working with TSO,
we find that all of the above memory models are effectively satisfied
(see
\url{https://en.cppreference.com/w/c/atomic/memory_order#Release-Acquire_ordering}
).

\subsection{Atomic Load Explicit}

Given atomic object type \texttt{A} and value type \texttt{C}:
\begin{nicec}
C atomic_load_explicit( const volatile A* obj, memory_order order )
\end{nicec}
then, with TSO, and \texttt{order} being relaxed, acquire or release,
we have:
\begin{nicec}
result = atomic_load_explicit( obj, order );
\end{nicec}
is equivalent to:
\begin{nicec}
atomic{ result = *obj  }
\end{nicec}

Usage:
\begin{nicec}
_Atomic_Load_uint( &gate->go_ahead, ATOMIC_ORDER_RELAXED )
_Atomic_Load_uint( &lock->now_serving, ATOMIC_ORDER_ACQUIRE )
_Atomic_Load_uint( &lock->now_serving, ATOMIC_ORDER_RELAXED )
_Atomic_Load_uint( &the_thread->Wait.flags, ATOMIC_ORDER_RELAXED )
_Atomic_Load_uint( &the_thread->Wait.flags, ATOMIC_ORDER_ACQUIRE )
\end{nicec}


\subsection{Store Load Explicit}

Given atomic object type \texttt{A} and value type \texttt{C}:
\begin{nicec}
void atomic_store_explicit( volatile A* obj, C desired, memory_order order )
\end{nicec}
then, with TSO, and \texttt{order} being relaxed, acquire or release,
we have:
\begin{nicec}
atomic_store_explicit( obj, desired, order );
\end{nicec}
is equivalent to:
\begin{nicec}
atomic{ *obj = desired }
\end{nicec}

Usage:
\begin{nicec}
_Atomic_Store_uint( &gate->go_ahead, 0, ATOMIC_ORDER_RELAXED )
_Atomic_Store_uint( &gate->go_ahead, 1, ATOMIC_ORDER_RELAXED )
_Atomic_Store_uint( &lock->now_serving, next_ticket, ATOMIC_ORDER_RELEASE )
_Atomic_Store_uint( &the_thread->Wait.flags, flags, ATOMIC_ORDER_RELAXED )
\end{nicec}

\subsection{Atomic Fetch-Add}

Given atomic object type \texttt{A}, value \texttt{C},
and value or pointer-difference type \texttt{M}:
\begin{nicec}
C atomic_fetch_add_explicit( volatile A* obj, M arg, memory_order order )
\end{nicec}
then, with TSO, and \texttt{order} being relaxed, acquire or release,
we have:
\begin{nicec}
result = atomic_fetch_add_explicit( obj, arg, order )
\end{nicec}
is equivalent to:
\begin{nicec}
atomic{ result = *obj ; *obj = *obj + arg }
\end{nicec}

Usage:
\begin{nicec}
_Atomic_Fetch_add_uint( &lock->next_ticket, 1U, ATOMIC_ORDER_RELAXED )
\end{nicec}
