\subsection{Chapter 12 Semaphore Manager}

\subsubsection{Introduction}

p206:

The semaphore manager utilizes standard Dijkstra counting semaphores
to provide synchronization and mutual exclusion capabilities.

\subsubsection{Background}

p207:

A binary semaphore (not a simple binary semaphore)
can be used to control access to a single resource.
\dots
In this instance, the semaphore would be created with an initial count of one to
indicate that no task is executing the critical section of code.

A binary semaphore must be released by the task that obtained it.

A counting semaphore can be used to control access
to a pool of two or more resources.
\dots
access to three printers could be administered by a semaphore
created with an initial count of three.

Task synchronization may be achieved by creating a semaphore
with an initial count of zero.

Simple binary semaphores do not allow nested access
and so can be used for task synchronization.

p208:

\verb"RTEMS_SIMPLE_BINARY_SEMAPHORE":
restrict values to 0 and 1,
do not allow nested access,
allow deletion of locked semaphore.

p209:

Some combinatinos\emph{(sic.)} of these attributes are invalid.

The following tree figure illustrates the valid combinations.

Classic Semaphore
$\mapsto \{$
 Counting Semaphore; Simple Binary; Binary (Mutex)
$\}$

Counting Semaphore
$\mapsto \{$
 FIFO; Priority
$\}$

Simple Binary
$\mapsto \{$
 FIFO; Priority
$\}$

Binary (Mutex)
$\mapsto \{$
 FIFO; Priority
$\}$

Binary (Mutex)
$\mapsto \{$
 Priority
 $\mapsto \{$
   No Protocol; Priority Inheritance Protocol; Priority Ceiling Protocol
 $\}$
$\}$

\subsubsection{Operations}

\paragraph{Creating a Semaphore}

p210:

If a binary semaphore is created with a count of zero (0)
to indicate that it has been allocated,
then the task creating the semaphore
is considered the current holder of the semaphore.

RTEMS allocates a Semaphore Control Block (SMCB) from the SMCB free list.

\paragraph{Obtaining Semaphore IDs}

\paragraph{Acquiring a Semaphore}

If the semaphore’s count is greater than zero
then decrement the semaphore’s count
else wait for release of semaphore then return SUCCESSFUL.

If the task blocked waiting for a binary semaphore
using priority inheritance
and the task’s priority is greater than that of
the task currently holding the semaphore,
then the holding task will inherit the priority of the blocking task.

When a task successfully obtains a semaphore using priority ceiling
and the priority ceiling for this semaphore is greater than that of the holder,
then the holder’s priority will be elevated.

\paragraph{Releasing a Semaphore}

p211:

If there are no tasks are waiting on this semaphore
then increment the semaphore’s count
else assign semaphore to a waiting task and return SUCCESSFUL.

If this is the outermost release of a binary semaphore
that uses priority inheritance or priority ceiling
and the task does not currently hold any other binary semaphores,
then the task performing the \verb"rtems_semaphore_release"
will have its priority restored to its normal value.

\paragraph{Deleting a Semaphore}

As a result of this directive,
all tasks blocked waiting to acquire the semaphore
will be readied
and returned a status code which indicates that the semaphore was deleted.
