\section{Pre/Post Techniques}

Here we present formalisations based on pre- and post-conditions,
most notably Hoare Triples and Weakest precondition.

\subsection{Hoare Triples}

Hoare triple $p \{Q\} r$ means that if program $Q$
starts running when $p$ is true,
then it terminates in a state where $r$ is true.

In UTP we can define Hoare triples as
$$
  p \{Q\} r  \defs  [Q \implies (p \implies r')]
$$

Proof rules for language constructs:
\begin{eqnarray*}
   &\vdash& r[e/x] \{ x:= e\} r
\\ (p\land b)\{P\}r , (p\land \lnot b)\{Q\}r
   &\vdash&
   p\{P \cond b Q\}r
\\ p\{P\}s , s\{Q\}r &\vdash& p\{P;Q\}r
\\ b \land c\{P\}c &\vdash& c\{b \whl P\}(\lnot b \land c)
\\ &\vdash& p \{Skip\} p
\\ p\{P\}r , p \{Q\} r &\vdash& p \{ P \ndc Q\} r
\end{eqnarray*}
Note---the following always hold: $false\{P\}r$ and $p\{P\}true$.


\newpage
\subsection{Weakest Precondition}

Weakest precondition $prog \WP post$
means the weakest pre-condition under which running \m{prog}
will guarantee outcome (condition) \m{post}.


In UTP we can define $\WP$ as
$$
  Q \WP r \defs \lnot(Q;\lnot r')
$$

It can be shown to obey the following laws:
\begin{eqnarray*}
      x:=e \WP r &\equiv& r[e/x]
   \\ P;Q \WP r &\equiv& P \WP ( Q \WP r)
   \\ (P \cond c Q) \WP r &\equiv& (P \WP r) \cond c (Q \WP r)
   \\ (P \ndc Q) \WP r &\equiv& (P \WP r) \land (Q \WP r)
\end{eqnarray*}

If we define an Invariant \m{inv} and variant \m V,
then we can define \m{c*P \WP r}.
The variant \m V\ is an expression defined over all the state variables
whose type has an associated well-founded ordering ($<$).
The most common type is is natural numbers under standard numeric order.

If we ignore termination, we can define the following
(the weakest liberal precondition)
\begin{eqnarray*}
      (c \whl P ) \WLP r
      &\defs&
      inv
    \\ && {} \land (c \land inv \implies P \WLP inv)
    \\ && {} \land (\lnot c \land inv \implies r)
\end{eqnarray*}

For WP, we need to say that the variant $V$ decreases w.r.t $<$
every time the loop body executes:
\begin{eqnarray*}
      (c \whl P ) \WP r
      &\defs&
      inv
    \\ && {} \land (c \land inv \implies P \WP (inv \land V' < V))
    \\ && {} \land (\lnot c \land inv \implies r)
\end{eqnarray*}

In practice, we use WLP,
and factor termination reasoning out as a separate step.



We can automatically put in placeholder names for \m{inv}
and \m V,
but at some stage the user will have to decide what they should be.

\newpage
\subsection{Verification Condition Generation}

An assertion \m{\assrt c} is a (specification)
statement that a condition holds.
\begin{eqnarray*}
   \assrt c &\defs& Skip \cond c Chaos
\\ \assrt b ; \assrt c &=& \assrt{(b \land c)}
\end{eqnarray*}

A program is appropriately annotated if
\begin{itemize}
  \item
    in a sequence \m{P_1;P_2;\cdots;P_n}
    there is an assertion before every statement that is
    \emph{not} an assignment or \m{Skip}.
  \item
    in every while-loop there is an (invariant) assertion
    at the start of the loop body
  \item
    The first assertion should be a consequence of
    the  pre-condition from the specification
  \item
    The last assertion should imply the specification post-condition.
  \item
    Note: this approach works for post-conditions that are \emph{conditions}
    (i.e. snapshots of state, and not before-after relations).
\end{itemize}

\begin{itemize}
  \item VCs are generated for all the program statements
  \item We shall define VC generation recursively over program structure
  \item
    Given that the last statement is an assignment
    \begin{eqnarray*}
          \lefteqn{genVC( \assrt p; P_1;\ldots;P_{n-1};v:=e;\assrt q)}
        \\ &=& genVC( p_\bot; P_1;\ldots;P_{n-1};q[e/v]_\bot)
    \end{eqnarray*}
    We drop the assignment and replace all free occurrences of \m v
    in the last assertion by \m e.
  \item
    Given that the last statement is an \emph{not} an assignment
    \begin{eqnarray*}
      \lefteqn{genVC( \assrt p; P_1;\ldots;P_{n-1};\assrt r;P_n;\assrt q)}
    \\ &=& genVC(\assrt r;P_n;\assrt q)
    \\ && {} \cup genVC( \assrt p; P_1;\ldots;P_{n-1};\assrt r)
    \end{eqnarray*}
    We process the last statement
    and the recursively treat the rest of the sequence.
\end{itemize}

\paragraph{VC generation (\m{:=})}

The pre-condition must imply the post-condition with \m v replaced by \m e.
$$genVC(\assrt p;x:=e;\assrt q) \defs \setof{ p \implies q[e/x] }$$


\paragraph{VC generation (\m{\cond{}})}

\begin{eqnarray*}
      \lefteqn{genVC(\assrt p; P_1 \cond c P_2 ; \assrt q)}
    \\ &=& genVC( \assrt{(p \land c)};P_1;\assrt q )
    \\ && {} \cup {}
    \\ && genVC( \assrt{(p \land \lnot c)};P_2;\assrt q )
\end{eqnarray*}

\paragraph{VC generation (\m{\whl})}

\begin{eqnarray*}
  \lefteqn{genVC( \assrt p ; c * ( \assrt i ; P ) ; \assrt q ) }
\\&=& \setof{ p \implies i, i \land \lnot c \implies q }
\\ && \cup
\\&& genVC( \assrt{(i \land c)}; P ; \assrt i)
\end{eqnarray*}

These correspond closely to our proof technique for loops:
\begin{itemize}
  \item \m{p \implies i}
   \\--- pre-condition sets up invariant
  \item \m{i \land \lnot c \implies q}
   \\--- invariant and termination satisfies postcondition
  \item \m{ \assrt{(i \land c)}; P ; \assrt i}
    \\--- invariant and loop-condition preserve the invariant
\end{itemize}
