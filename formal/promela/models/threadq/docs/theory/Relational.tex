\section{Relational Techniques}

Here we present a formalisation based on relations between starting
and finishing states.

\subsection{Unifying Theories of Programming}

We use un-dashed variable $x$ ($s$)
to represent the values of program variable $x$ (program state $s$)
before the program runs.
We use dashed variable $x'$ ($s'$)
to represent the values of program variable $x$ (program state $s$)
after the program terminates.


Partial correctness semantics:
\begin{eqnarray*}
   x := e      &\defs& x' = e \land \nu' = \nu
\\ Skip        &\defs& \nu' = \nu
\\ P ; Q       &\defs& \exists \nu_0 \bullet P[\nu_0/\nu'] \land Q[\nu_0/\nu]
\\ P \cond c Q &\defs& c \land P \lor \lnot c \land Q
\\ c \whl P    &\defs& \mu L \bullet (P;L \cond c Skip)
\\ P \ndc Q    &\defs& P \lor Q
\end{eqnarray*}
Here $\nu$ and $\nu'$ refer to all variables not explicitly
mentioned in the rule.


For total correctness we need to
use special variable $ok:\Bool$ to assert that a program has started properly,
and $ok':\Bool$ to assert that it has terminated.
