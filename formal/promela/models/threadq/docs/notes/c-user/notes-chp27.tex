\subsection{Chapter 27 SYMMETRIC MULTIPROCESSING (SMP)}

\subsubsection{Introduction}

p526:

The RTEMS interpretation of real-time on SMP
is the support for Clustered Scheduling
with priority based schedulers and adequate locking protocols.

\subsubsection{Background}

pp527--8:

In an SMP system with N processors,
these are the new execution characteristics.
\begin{itemize}
  \item N tasks execute in parallel
  \item hardware events result in interrupts
\end{itemize}
There is true parallelism with a task executing on each processor
and the possibility of interrupts occurring on each processor.
Thus in contrast to their being one task and one interrupt
to consider on a uniprocessor,
there are N tasks and potentially N simultaneous interrupts
to consider on an SMP system.

p528:

Affinity is used to specify the subset of processors
in an SMP system on which a particular task can execute.

Only a subset of the available schedulers support affinity.

The scheduler with support for arbitary processor affinities
uses a proof of concept implementation.
See \url{https://devel.rtems.org/ticket/2510}
(this ticket is worth re-reading at some stage!).

With more than one processor in the system tasks can migrate from one processor to another.
There are four reasons why tasks migrate in RTEMS.
\begin{itemize}
  \item
    The scheduler changes explicitly
    via \verb"rtems_task_set_scheduler()" or similar directives.
  \item
    The task processor affinity changes explicitly
    via \verb"rtems_task_set_affinity()" or similar directives.
  \item
    The task resumes execution after a blocking operation.
    On a priority based scheduler it will evict
    the lowest priority task
    currently assigned to a processor in the processor set
    managed by the scheduler instance.
  \item
    The task moves temporarily to another scheduler instance
    due to locking protocols like
    the \emph{Multiprocessor Resource Sharing Protocol (MrsP)}
    or the \emph{O(m) Independence-Preserving Protocol (OMIP)}.
\end{itemize}

p529:

We have clustered scheduling in case
the set of processors of a system is partitioned
into nonempty pairwise-disjoint subsets of processors.
These subsets are called clusters.
Clusters with a cardinality of one are partitions.
Each cluster is owned by exactly one scheduler instance.
In case the cluster size equals the processor count,
it is called global scheduling.

Clustered scheduling was implemented for RTEMS SMP
to best use the cache topology of a system
and to keep the worst-case latencies under control.
The low-level SMP locks use FIFO ordering.

The problem is to provide synchronization primitives
for inter-cluster synchronization
(more than one cluster is involved in the synchronization process).
In RTEMS there are currently some means available
\begin{itemize}
  \item events,
  \item message queues,
  \item mutexes using the \emph{O(m) Independence-Preserving Protocol (OMIP)},
  \item
    mutexes using the \emph{Multiprocessor Resource Sharing Protocol (MrsP)},
    and
  \item binary and counting semaphores.
\end{itemize}
The clustered scheduling approach enables separation of functions
with real-time requirements
and functions that profit from fairness and high throughput
provided the scheduler instances are fully decoupled
and adequate inter-cluster synchronization primitives are used.

p531:

There is no public RTEMS API for atomic operations.
It is recommended to use the standard C
\href{https://en.cppreference.com/w/c/atomic}{$<$stdatomic.h$>$}
or C++ \href{https://en.cppreference.com/w/cpp/atomic/atomic}{$<$atomic$>$}
APIs in applications.

\subsubsection{Application Issues}

p532:

27.3.3 Disabling of Thread Preemption
A thread which disables preemption prevents
that a higher priority thread gets hold of its processor involuntarily.
In uniprocessor configurations,
this can be used to ensure mutual exclusion at thread level.
In SMP configurations,
however,
more than one executing thread may exist.
Thus,
it is impossible to ensure mutual exclusion using this mechanism.
In order to prevent that applications using preemption for this purpose,
would show inappropriate behaviour,
this feature is disabled in SMP configurations
and its use would case(\emph{s.i.c.}) run-time errors.

p533:

27.3.4 Disabling of Interrupts
In SMP configurations,
however,
disabling the interrupts on one processor has no effect on other processors.
So,
this is insufficient to ensure system-wide mutual exclusion.
The macros
\begin{itemize}
  \item \verb"rtems_interrupt_disable()",
  \item \verb"rtems_interrupt_enable()", and
  \item \verb"rtems_interrupt_flash()".
\end{itemize}
are disabled in SMP configurations
and its use will cause compile-time warnings and link-time errors.
In the unlikely case that interrupts must be disabled on the current processor,
the
\begin{itemize}
  \item \verb"rtems_interrupt_local_disable()", and
  \item \verb"rtems_interrupt_local_enable()".
\end{itemize}
macros are now available in all configurations.

A new low-level synchronization primitive was added – interrupt locks.
The interrupt locks are a simple API layer
on top of the SMP locks used for low-level synchronization
in the operating system core.
Currently,
they are implemented as a ticket lock.
In uniprocessor configurations,
they degenerate to simple interrupt disable/enable sequences
by means of the C pre-processor.
It is disallowed to acquire a single interrupt lock in a nested way.
This will result in an infinite loop with interrupts disabled.
While converting legacy code to interrupt locks,
care must be taken to avoid this situation to happen.

p534:

In contrast to POSIX spinlock implementation on Linux or FreeBSD,
it is not allowed to call blocking operating system services
inside the critical section.
A recursive lock attempt is a severe usage error resulting in an infinite loop
with interrupts disabled.
Nesting of different locks is allowed.
The user must ensure that no deadlock can occur.

Interrupt service routines must take this into account
and use proper locking mechanisms
to protect critical sections
from interference by threads
(interrupt locks or POSIX spinlocks).

In SMP configurations,
the timer service routine may already run
and wait on an SMP lock owned by the thread which is about to stop the timer.
This opens the door to subtle synchronization issues.
During destruction of objects,
special care must be taken to ensure that
timer service routines cannot access (partly or fully) destroyed objects.

27.3.7 False Sharing of Cache Lines Due to Objects Table
\dots
High-performance SMP applications need full control of the object storage.
Therefore,
self-contained synchronization objects are now available for RTEMS.

\subsubsection{Implementation Details}

p535:

\emph{27.4.1 Low-Level Synchronization}

All low-level synchronization primitives are implemented
using C11 atomic operations.

Four synchronization primitives are currently available
\begin{itemize}
  \item ticket locks (mutual exclusion),
  \item MCS locks (mutual exclusion),
  \item barriers, implemented as a sense barrier, and
  \item sequence locks
\end{itemize}

A vital requirement for low-level mutual exclusion is FIFO fairness
since we are interested in a predictable system and not maximum throughput.

p536:

\emph{27.4.2 Internal Locking}

In SMP configurations, the operating system uses non-recursive SMP locks for low-level mutual
exclusion. The locking domains are roughly
\begin{itemize}
  \item a particular data structure,
  \item the thread queue operations,
  \item the thread state changes, and
  \item the scheduler operations.
\end{itemize}

For a good average-case performance it is vital that every high-level
synchronization object,
e.g. mutex,
has its own SMP lock.
In the average-case,
only this SMP lock should be involved to carry out a specific operation,
e.g. obtain/release a mutex.
In general,
the high-level synchronization objects have a thread queue embedded and use its
SMP lock.

In case a thread must block on a thread queue,
then things get complicated.
The executing thread first acquires the SMP lock of the thread queue
and then figures out that it needs to block.
The procedure to block the thread on this particular thread queue
involves state changes of the thread itself
and for this thread-specific SMP locks must be used.

In order to determine if a thread is blocked on a thread queue or not
thread-specific SMP locks must be used.
A thread priority change must propagate this to the thread queue
(possibly recursively).
Care must be taken to not have a lock order reversal between
thread queue and thread-specific SMP locks.

Each scheduler instance has its own SMP lock.
For the scheduler helping protocol
multiple scheduler instances may be in charge of a thread.
It is not possible to acquire two scheduler instance SMP locks at the same time,
otherwise deadlocks would happen.
A thread-specific SMP lock is used to synchronize the thread data
shared by different scheduler instances.
The thread state SMP lock protects various things,
e.g. the thread state,
join operations,
signals,
post-switch actions,
the home scheduler instance, etc.

p538:

\emph{27.4.4 Scheduler Helping Protocol}

Each thread has a scheduler node for each scheduler instance in
the system which are located in its TCB.
A thread has exactly one home scheduler instance
which is set during thread creation.

Due to the locking protocols a thread may gain access
to scheduler nodes of other scheduler instances.
This allows the thread to temporarily migrate
to another scheduler instance in case of preemption.

For the scheduler helping protocol the following
operations must be implemented by an SMP-aware scheduler
\begin{itemize}
  \item ask a scheduler node for help,
  \item reconsider the help request of a scheduler node,
  \item withdraw a schedule node.
\end{itemize}

In case a thread is allowed to use more than one scheduler node
it will ask these nodes for help
\begin{itemize}
  \item in case of preemption, or
  \item an unblock did not schedule the thread, or
  \item a yield was successful.
\end{itemize}

The actual ask for help scheduler operations are carried out
as a side-effect of the thread dispatch procedure.

After a thread dispatch
the reconsider help request operation is used
to clean up stale help registrations in the scheduler contexts.

p539:

\emph{27.4.5 Thread Dispatch Details}

The thread context is protected by a TTAS lock
embedded in the context to ensure that
it is used on at most one processor at a time.
\dots
This implementation turned out to be quite efficient
and no lock contention was observed in the testsuite.

The context-switch is performed with interrupts enabled.
During the transition from the executing to the heir thread
neither the stack of the executing nor the heir thread must be used
during interrupt processing.
For this purpose a temporary per-processor stack is set up
which may be used by the interrupt prologue
before the stack is switched to the interrupt stack.

p540:

\emph{27.4.7 Thread Pinning}

Thread pinning ensures that a thread is only dispatched to the processor
on which it is pinned.
It may be used to access per-processor data structures
in critical sections with enabled thread dispatching,
e.g. a pinned thread is allowed to block.

Thread pinning must be used with care,
since it prevents help through the locking protocols.
This makes the \textit{OMIP} and \textit{MrsP} locking protocols ineffective 
if pinned threads are involved.
